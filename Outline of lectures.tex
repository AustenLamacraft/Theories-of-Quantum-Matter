\section{Outline of lectures}

16 90 minute lectures. \verb|This denotes PyCav integration.|

\subsection{Many body wavefunctions and statistics}

Examples of the 1D Fermi gas, Calogero--Sutherland model, and QHE. \verb|Monte Carlo simulation of Laughlin state. Statistics of Bose condensation|

\subsection{Lieb--Liniger model}

Bethe's wave function. Excited states and the classical limit. Inpenetrable limit; attractive ground state (analytical form) \verb|Solving the Bethe equations|

\subsection{Collective excitations}

Phonons. Quantizing a chain. Hydrodynamic picture and Luttinger liquid. Jastrow form of phonon ground state wavefunction in terms of particles. Feynman–Bijl formula as variational wavefunction.

\subsection{Spin chains}

Simple models of magnetism: Heisenberg model. \verb|Exact diagonalization of spin chains.| Magnons. Mention Bethe ansatz solution. Relation to SEP. \verb|Spin waves and solitons.|
Antiferromagnetic order. Symmetry breaking. Anderson tower. Other examples: polar condensate.

\subsection{Second quantization}

Fock states and occupation numbers. Creation and annihilation operators. The case of fermions. Representation of operators. \textit{Possible pedagogical approach: state conditions for CCR and CAR algebras and let students figure out correspondence}

Examples of Schwinger bosons. Their classical version

\subsection{More second quantization}

Density matrix. Density correlations. Second quantized Hamiltonians. Momentum space form. Hartree--Fock theory. \emph{When perturbations attack I}. Application: Matveev--Yue--Glazman, Coulomb Blockade. 

\subsection{Response and correlation}

Response functions. Structure factor. Lindhard and dielectric function. Sum rules.

\subsection{Jellium}

Hubbard--Stratonovich treatment of RPA. Recover dielectric function. How do we get at the functional determinant in the simplest way? Perhaps just be expanding the density in terms of the Coulomb field -- this makes the closest connection to a response function.

\subsection{Fermi gas}

Trial wavefunction describing quasiparticles. Quasiparticle residue. Landau Fermi liquid via perturbation theory. 

\subsection{Superconductivity}

Heisenberg's problem as a first encounter with the idea of renormalization. \emph{When perturbations attack II}. Cooper divergences. BCS theory. Anderson spin picture.

\subsection{The Kondo effect}

Log singular renormalization of scattering at second order. Poor man's scaling by Schrieffer--Wolf. \verb|Numerical renormalization group| See \href{https://github.com/LucasNogueiraMartins/NRG-Didactic}{this GitHub repository}

\subsection{Bose gas}

Gross--Pitaevskii. \emph{When perturbations attack III}. Bogoliubov theory and superfluidity. \verb|Gross--Pitaevskii in traps. Vortex lattices.|


\subsection{Dissipation in Quantum Mechanics}

Orthogonality catastrophe. Caldeira--Leggett environment.

\subsection{Lattice models and strong correlations}

Tight binding. Hubbard model and Mott transition. Bose--Hubbard. Superexchange via Schrieffer--Wolf.

\subsection{Jordan--Wigner}

Jordan--Wigner treatment of the XY model. Lieb--Schultz--Mattis theorem.


\subsection{The Ising Model and Quantum Phase Transitions}

1D classical Ising doesn't order. Transverse Ising as quantum phase transition. Transfer matrix and relation to 2d classical Ising. \verb|Classical Ising simulation.| Majorana edge modes.


% \section{Lent minor option}


% \begin{enumerate}

% \item Topological defects. Vortices, Skyrmions. 

% \item Berry phase. Magnetic monopoles. Topological Band structure.

% \item Kitaev model?

% \item Fractional statistics?
% \end{enumerate}